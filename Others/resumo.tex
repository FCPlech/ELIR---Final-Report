\begin{thesisresumo}
A garantia da disponibilidade de energia elétrica é um dos maiores problemas enfrentados pelas concessionárias de energia, pois além do custo elevado, a manutenção das linhas de transmissão é uma tarefa de alto risco para atividade humana. Atualmente, a inspeção é realizada a partir de aeronoves tripuladas ou com técnicos que apoiados a linha executam as tarefas com uma vestimenta especial. A fim de reduzir o trabalho humano em atividades de risco muitos alternativas utilizando a robótica e o monitoramente térmico estão sendo utilizadas. O presente trabalho busca descrever os procedimentos realizados para o desenvolvimento de um sistema de percepção para um robo de inspeção de linhas de alta tensão. 

\ \\

% use de três a cinco palavras-chave

\textbf{Palavras-chave}: Linhas de transmissão, Inspeção, Robótica, Pontos quentes

\end{thesisresumo}
